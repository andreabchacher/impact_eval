% Options for packages loaded elsewhere
\PassOptionsToPackage{unicode}{hyperref}
\PassOptionsToPackage{hyphens}{url}
%
\documentclass[
  ignorenonframetext,
]{beamer}
\usepackage{pgfpages}
\setbeamertemplate{caption}[numbered]
\setbeamertemplate{caption label separator}{: }
\setbeamercolor{caption name}{fg=normal text.fg}
\beamertemplatenavigationsymbolsempty
% Prevent slide breaks in the middle of a paragraph
\widowpenalties 1 10000
\raggedbottom
\setbeamertemplate{part page}{
  \centering
  \begin{beamercolorbox}[sep=16pt,center]{part title}
    \usebeamerfont{part title}\insertpart\par
  \end{beamercolorbox}
}
\setbeamertemplate{section page}{
  \centering
  \begin{beamercolorbox}[sep=12pt,center]{part title}
    \usebeamerfont{section title}\insertsection\par
  \end{beamercolorbox}
}
\setbeamertemplate{subsection page}{
  \centering
  \begin{beamercolorbox}[sep=8pt,center]{part title}
    \usebeamerfont{subsection title}\insertsubsection\par
  \end{beamercolorbox}
}
\AtBeginPart{
  \frame{\partpage}
}
\AtBeginSection{
  \ifbibliography
  \else
    \frame{\sectionpage}
  \fi
}
\AtBeginSubsection{
  \frame{\subsectionpage}
}
\usepackage{amsmath,amssymb}
\usepackage{iftex}
\ifPDFTeX
  \usepackage[T1]{fontenc}
  \usepackage[utf8]{inputenc}
  \usepackage{textcomp} % provide euro and other symbols
\else % if luatex or xetex
  \usepackage{unicode-math} % this also loads fontspec
  \defaultfontfeatures{Scale=MatchLowercase}
  \defaultfontfeatures[\rmfamily]{Ligatures=TeX,Scale=1}
\fi
\usepackage{lmodern}
\ifPDFTeX\else
  % xetex/luatex font selection
\fi
% Use upquote if available, for straight quotes in verbatim environments
\IfFileExists{upquote.sty}{\usepackage{upquote}}{}
\IfFileExists{microtype.sty}{% use microtype if available
  \usepackage[]{microtype}
  \UseMicrotypeSet[protrusion]{basicmath} % disable protrusion for tt fonts
}{}
\makeatletter
\@ifundefined{KOMAClassName}{% if non-KOMA class
  \IfFileExists{parskip.sty}{%
    \usepackage{parskip}
  }{% else
    \setlength{\parindent}{0pt}
    \setlength{\parskip}{6pt plus 2pt minus 1pt}}
}{% if KOMA class
  \KOMAoptions{parskip=half}}
\makeatother
\usepackage{xcolor}
\newif\ifbibliography
\usepackage{color}
\usepackage{fancyvrb}
\newcommand{\VerbBar}{|}
\newcommand{\VERB}{\Verb[commandchars=\\\{\}]}
\DefineVerbatimEnvironment{Highlighting}{Verbatim}{commandchars=\\\{\}}
% Add ',fontsize=\small' for more characters per line
\usepackage{framed}
\definecolor{shadecolor}{RGB}{248,248,248}
\newenvironment{Shaded}{\begin{snugshade}}{\end{snugshade}}
\newcommand{\AlertTok}[1]{\textcolor[rgb]{0.94,0.16,0.16}{#1}}
\newcommand{\AnnotationTok}[1]{\textcolor[rgb]{0.56,0.35,0.01}{\textbf{\textit{#1}}}}
\newcommand{\AttributeTok}[1]{\textcolor[rgb]{0.13,0.29,0.53}{#1}}
\newcommand{\BaseNTok}[1]{\textcolor[rgb]{0.00,0.00,0.81}{#1}}
\newcommand{\BuiltInTok}[1]{#1}
\newcommand{\CharTok}[1]{\textcolor[rgb]{0.31,0.60,0.02}{#1}}
\newcommand{\CommentTok}[1]{\textcolor[rgb]{0.56,0.35,0.01}{\textit{#1}}}
\newcommand{\CommentVarTok}[1]{\textcolor[rgb]{0.56,0.35,0.01}{\textbf{\textit{#1}}}}
\newcommand{\ConstantTok}[1]{\textcolor[rgb]{0.56,0.35,0.01}{#1}}
\newcommand{\ControlFlowTok}[1]{\textcolor[rgb]{0.13,0.29,0.53}{\textbf{#1}}}
\newcommand{\DataTypeTok}[1]{\textcolor[rgb]{0.13,0.29,0.53}{#1}}
\newcommand{\DecValTok}[1]{\textcolor[rgb]{0.00,0.00,0.81}{#1}}
\newcommand{\DocumentationTok}[1]{\textcolor[rgb]{0.56,0.35,0.01}{\textbf{\textit{#1}}}}
\newcommand{\ErrorTok}[1]{\textcolor[rgb]{0.64,0.00,0.00}{\textbf{#1}}}
\newcommand{\ExtensionTok}[1]{#1}
\newcommand{\FloatTok}[1]{\textcolor[rgb]{0.00,0.00,0.81}{#1}}
\newcommand{\FunctionTok}[1]{\textcolor[rgb]{0.13,0.29,0.53}{\textbf{#1}}}
\newcommand{\ImportTok}[1]{#1}
\newcommand{\InformationTok}[1]{\textcolor[rgb]{0.56,0.35,0.01}{\textbf{\textit{#1}}}}
\newcommand{\KeywordTok}[1]{\textcolor[rgb]{0.13,0.29,0.53}{\textbf{#1}}}
\newcommand{\NormalTok}[1]{#1}
\newcommand{\OperatorTok}[1]{\textcolor[rgb]{0.81,0.36,0.00}{\textbf{#1}}}
\newcommand{\OtherTok}[1]{\textcolor[rgb]{0.56,0.35,0.01}{#1}}
\newcommand{\PreprocessorTok}[1]{\textcolor[rgb]{0.56,0.35,0.01}{\textit{#1}}}
\newcommand{\RegionMarkerTok}[1]{#1}
\newcommand{\SpecialCharTok}[1]{\textcolor[rgb]{0.81,0.36,0.00}{\textbf{#1}}}
\newcommand{\SpecialStringTok}[1]{\textcolor[rgb]{0.31,0.60,0.02}{#1}}
\newcommand{\StringTok}[1]{\textcolor[rgb]{0.31,0.60,0.02}{#1}}
\newcommand{\VariableTok}[1]{\textcolor[rgb]{0.00,0.00,0.00}{#1}}
\newcommand{\VerbatimStringTok}[1]{\textcolor[rgb]{0.31,0.60,0.02}{#1}}
\newcommand{\WarningTok}[1]{\textcolor[rgb]{0.56,0.35,0.01}{\textbf{\textit{#1}}}}
\setlength{\emergencystretch}{3em} % prevent overfull lines
\providecommand{\tightlist}{%
  \setlength{\itemsep}{0pt}\setlength{\parskip}{0pt}}
\setcounter{secnumdepth}{-\maxdimen} % remove section numbering
\usepackage{bookmark}
\IfFileExists{xurl.sty}{\usepackage{xurl}}{} % add URL line breaks if available
\urlstyle{same}
\hypersetup{
  hidelinks,
  pdfcreator={LaTeX via pandoc}}

\author{}
\date{\vspace{-2.5em}}

\begin{document}

\begin{frame}

Lecture 3: Causality
\end{frame}

\begin{frame}{Causality}
\phantomsection\label{causality}
We want to know ``Does X cause Y?'' and ``How much does X cause Y?'' We
often want to do this while only having access to observational data
This is what the class is about
\end{frame}

\begin{frame}{Why Causality?}
\phantomsection\label{why-causality}
\begin{itemize}
\tightlist
\item
  Many of the interesting questions we might want to answer with data
  are causal
\item
  Some are non-causal, too - for example, ``how can we predict whether
  this photo is of a dog or a cat'' is vital to how Google Images works,
  but it doesn't care what \emph{caused} the photo to be of a dog or a
  cat
\item
  Nearly every \emph{why} question is causal
\item
  And when we're talking about people, \emph{why} is often what we want
  to know!
\end{itemize}
\end{frame}

\begin{frame}{Also}
\phantomsection\label{also}
\begin{itemize}
\tightlist
\item
  This is economists' comparative advantage!
\item
  Plenty of fields do statistics. But very few make it standard training
  for their students to understand causality
\item
  This understanding of causality makes economists very useful!
  \emph{This} is one big reason why tech companies have whole economics
  departments in them
\end{itemize}
\end{frame}

\begin{frame}{Bringing us to\ldots{}}
\phantomsection\label{bringing-us-to}
\begin{itemize}
\tightlist
\item
  Part of this half of the class will be understanding what causality
  \emph{is} and how we can find it
\item
  Another big part will be understanding common \emph{research designs}
  for uncovering causality in data when we can't do an experiment
\item
  These, more than supply \& demand, more than ISLM, are the tools of
  the modern economist!
\end{itemize}
\end{frame}

\begin{frame}[fragile]{So what is causality?}
\phantomsection\label{so-what-is-causality}
\begin{itemize}
\tightlist
\item
  We say that \texttt{X} \emph{causes} \texttt{Y} if\ldots{}
\item
  were we to intervene and \emph{change} the value of \texttt{X} without
  changing anything else\ldots{}
\item
  then \texttt{Y} would also change as a result
\end{itemize}
\end{frame}

\begin{frame}{Some examples}
\phantomsection\label{some-examples}
Examples of causal relationships!

Some obvious:

\begin{itemize}
\tightlist
\item
  A light switch being set to on causes the light to be on
\item
  Setting off fireworks raises the noise level
\end{itemize}

Some less obvious:

\begin{itemize}
\tightlist
\item
  Getting a college degree increases your earnings
\item
  Tariffs reduce the amount of trade
\end{itemize}
\end{frame}

\begin{frame}{Some examples}
\phantomsection\label{some-examples-1}
Examples of non-zero \emph{correlations} that are not \emph{causal} (or
may be causal in the wrong direction!)

Some obvious:

\begin{itemize}
\tightlist
\item
  People tend to wear shorts on days when ice cream trucks are out
\item
  Rooster crowing sounds are followed closely by sunrise*
\end{itemize}

Some less obvious:

\begin{itemize}
\tightlist
\item
  Colds tend to clear up a few days after you take Emergen-C
\item
  The performance of the economy tends to be lower or higher depending
  on the president's political party
\end{itemize}

*This case of mistaken causality is the basis of the film Rock-a-Doodle
which I remember being very entertaining when I was six.
\end{frame}

\begin{frame}{Important Note}
\phantomsection\label{important-note}
\begin{itemize}
\tightlist
\item
  ``X causes Y'' \emph{doesn't} mean that X is necessarily the
  \emph{only} thing that causes Y
\item
  And it \emph{doesn't} mean that all Y must be X
\item
  For example, using a light switch causes the light to go on
\item
  But not if the bulb is burned out (no Y, despite X), or if the light
  was already on (Y without X), and it ALSO needs electicity (something
  else causes Y)
\item
  But still we'd say that using the switch causes the light! The
  important thing is that X \emph{changes the distribution} of Y, not
  that it necessarily makes it happen for certain
\end{itemize}
\end{frame}

\begin{frame}{So How Can We Tell?}
\phantomsection\label{so-how-can-we-tell}
\begin{itemize}
\tightlist
\item
  As just shown, there are plenty of \emph{correlations} that aren't
  \emph{causal}
\item
  So if we have a correlation, how can we tell if it \emph{is}?
\item
  For this we're going to have to think hard about \emph{causal
  inference}. That is, inferring causality from data
\end{itemize}
\end{frame}

\begin{frame}[fragile]{The Problem of Causal Inference}
\phantomsection\label{the-problem-of-causal-inference}
\begin{itemize}
\tightlist
\item
  Let's try to think about whether some \texttt{X} causes \texttt{Y}
\item
  That is, if we manipulated \texttt{X}, then \texttt{Y} would change as
  a result
\item
  For simplicity, let's assume that \texttt{X} is either 1 or 0, like
  ``got a medical treatment'' or ``didn't''
\end{itemize}
\end{frame}

\begin{frame}[fragile]{The Problem of Causal Inference}
\phantomsection\label{the-problem-of-causal-inference-1}
\begin{itemize}
\tightlist
\item
  Now, how can we know \emph{what would happen} if we manipulated
  \texttt{X}?
\item
  Let's consider just one person - Angela. We could just check what
  Angela's \texttt{Y} is when we make \texttt{X=0}, and then check what
  Angela's \texttt{Y} is again when we make \texttt{X=1}.
\item
  Are those two \texttt{Y}s different? If so, \texttt{X} causes
  \texttt{Y}!
\item
  Do that same process for everyone in your sample and you know in
  general what the effect of \texttt{X} on \texttt{Y} is
\end{itemize}
\end{frame}

\begin{frame}[fragile]{The Problem of Causal Inference}
\phantomsection\label{the-problem-of-causal-inference-2}
\begin{itemize}
\tightlist
\item
  You may have spotted the problem
\item
  Just like you can't be in two places at once, Angela can't exist both
  with \texttt{X=0} and with \texttt{X=1}. She either got that medical
  treatment or she didn't.
\item
  Let's say she did. So for Angela, \texttt{X=1} and, let's say,
  \texttt{Y=10}.
\item
  The other one, what \texttt{Y} \emph{would have been} if we made
  \texttt{X=0}, is \emph{missing}. We don't know what it is! Could also
  be \texttt{Y=10}. Could be \texttt{Y=9}. Could be \texttt{Y=1000}!
\end{itemize}
\end{frame}

\begin{frame}[fragile]{The Problem of Causal Inference}
\phantomsection\label{the-problem-of-causal-inference-3}
\begin{itemize}
\tightlist
\item
  Well, why don't we just take someone who actually DOES have
  \texttt{X=0} and compare their \texttt{Y}?
\item
  Because there are lots of reasons their \texttt{Y} could be different
  BESIDES \texttt{X}.
\item
  They're not Angela! A character flaw to be sure.
\item
  So if we find someone, Gareth, with \texttt{X=0} and they have
  \texttt{Y=9}, is that because \texttt{X} increases \texttt{Y}, or is
  that just because Angela and Gareth would have had different
  \texttt{Y}s anyway?
\end{itemize}
\end{frame}

\begin{frame}[fragile]{The Problem of Causal Inference}
\phantomsection\label{the-problem-of-causal-inference-4}
\begin{itemize}
\tightlist
\item
  The main goal we have in doing causal inference is in making \emph{as
  good a guess as possible} as to what that \texttt{Y} \emph{would have
  been} if \texttt{X} had been different
\item
  That ``would have been'' is called a \emph{counterfactual} - counter
  to the fact of what actually happened
\item
  In doing so, we want to think about two people/firms/countries that
  are basically \emph{exactly the same} except that one has \texttt{X=0}
  and one has \texttt{X=1}
\end{itemize}
\end{frame}

\begin{frame}{Potential Outcomes}
\phantomsection\label{potential-outcomes}
\begin{itemize}
\tightlist
\item
  The logic we just went through is the basis of the \emph{potential
  outcomes model}, which is one way of thinking about causality
\item
  It's not the only one, or the one we'll be mainly using, but it helps!
\item
  We can't observe the counterfactual, and must make an estimate of what
  the \emph{outcome} would \emph{potentially} have been under the
  counterfactual
\item
  Figuring out that makes a good counterfactual estimate is a key part
  of causal inference!
\end{itemize}
\end{frame}

\begin{frame}[fragile]{Experiments}
\phantomsection\label{experiments}
\begin{itemize}
\tightlist
\item
  A common way to do causal inference in many fields is an
  \emph{experiment}
\item
  If you can \emph{randomly assign} \texttt{X}, then you know that the
  people with \texttt{X=0} are, on average, exactly the same as the
  people with \texttt{X=1}
\item
  So that's an easy comparison!
\end{itemize}
\end{frame}

\begin{frame}{Experiments}
\phantomsection\label{experiments-1}
\begin{itemize}
\tightlist
\item
  When we're working with people/firms/countries, running experiments is
  often infeasible, impossible, or unethical
\item
  So we have to think hard about a \emph{model} of what the world looks
  like
\item
  So that we can use our model to figure out what the
  \emph{counterfactual} would be
\end{itemize}
\end{frame}

\begin{frame}{Models}
\phantomsection\label{models}
\begin{itemize}
\tightlist
\item
  In causal inference, the \emph{model} is our idea of what we think the
  process is that \emph{generated the data}
\item
  We have to make some assumptions about what this is!
\item
  We put together what we know about the world with assumptions and end
  up with our model
\item
  The model can then tell us what kinds of things could give us wrong
  results so we can fix them and get the right counterfactual
\end{itemize}
\end{frame}

\begin{frame}{Models}
\phantomsection\label{models-1}
\begin{itemize}
\tightlist
\item
  Wouldn't it be nice to not have to make assumptions?
\item
  Yeah, but it's impossible to skip!
\item
  We're trying to predict something that hasn't happened - a
  counterfactual
\item
  This is literally impossible to do if you don't have some model of how
  the data is generated
\item
  You can't even predict the sun will rise tomorrow without a model!
\item
  If you think you can, you're just don't realize the model you're using
  - that's dangerous!
\end{itemize}
\end{frame}

\begin{frame}[fragile]{An Example}
\phantomsection\label{an-example}
\begin{itemize}
\tightlist
\item
  Let's cheat again and know how our data is generated!
\item
  Let's say that getting \texttt{X} causes \texttt{Y} to increase by 1
\item
  And let's run a randomized experiment of who actually gets X
\end{itemize}

\begin{Shaded}
\begin{Highlighting}[]
\NormalTok{df }\OtherTok{\textless{}{-}} \FunctionTok{data.frame}\NormalTok{(}\AttributeTok{Y.without.X =} \FunctionTok{rnorm}\NormalTok{(}\DecValTok{1000}\NormalTok{),}\AttributeTok{X=}\FunctionTok{sample}\NormalTok{(}\FunctionTok{c}\NormalTok{(}\DecValTok{0}\NormalTok{,}\DecValTok{1}\NormalTok{),}\DecValTok{1000}\NormalTok{,}\AttributeTok{replace=}\NormalTok{T)) }\SpecialCharTok{\%\textgreater{}\%}
\FunctionTok{mutate}\NormalTok{(}\AttributeTok{Y.with.X =}\NormalTok{ Y.without.X }\SpecialCharTok{+} \DecValTok{1}\NormalTok{) }\SpecialCharTok{\%\textgreater{}\%}
\CommentTok{\#Now assign who actually gets X}
\FunctionTok{mutate}\NormalTok{(}\AttributeTok{Observed.Y =} \FunctionTok{ifelse}\NormalTok{(X}\SpecialCharTok{==}\DecValTok{1}\NormalTok{,Y.with.X,Y.without.X))}
\CommentTok{\#And see what effect our experiment suggests X has on Y}
\NormalTok{df }\SpecialCharTok{\%\textgreater{}\%} \FunctionTok{group\_by}\NormalTok{(X) }\SpecialCharTok{\%\textgreater{}\%} \FunctionTok{summarize}\NormalTok{(}\AttributeTok{Y =} \FunctionTok{mean}\NormalTok{(Observed.Y))}
\end{Highlighting}
\end{Shaded}

\begin{verbatim}
## # A tibble: 2 x 2
##       X      Y
##   <dbl>  <dbl>
## 1     0 0.0349
## 2     1 0.955
\end{verbatim}
\end{frame}

\begin{frame}[fragile]{An Example}
\phantomsection\label{an-example-1}
\begin{itemize}
\tightlist
\item
  Now this time we can't randomize X.
\end{itemize}

\begin{Shaded}
\begin{Highlighting}[]
\NormalTok{df }\OtherTok{\textless{}{-}} \FunctionTok{data.frame}\NormalTok{(}\AttributeTok{Z =} \FunctionTok{runif}\NormalTok{(}\DecValTok{10000}\NormalTok{)) }\SpecialCharTok{\%\textgreater{}\%} \FunctionTok{mutate}\NormalTok{(}\AttributeTok{Y.without.X =} \FunctionTok{rnorm}\NormalTok{(}\DecValTok{10000}\NormalTok{) }\SpecialCharTok{+}\NormalTok{ Z, }\AttributeTok{Y.with.X =}\NormalTok{ Y.without.X }\SpecialCharTok{+} \DecValTok{1}\NormalTok{) }\SpecialCharTok{\%\textgreater{}\%}
  \CommentTok{\#Now assign who actually gets X}
  \FunctionTok{mutate}\NormalTok{(}\AttributeTok{X =}\NormalTok{ Z }\SpecialCharTok{\textgreater{}}\NormalTok{ .}\DecValTok{7}\NormalTok{,}\AttributeTok{Observed.Y =} \FunctionTok{ifelse}\NormalTok{(X}\SpecialCharTok{==}\DecValTok{1}\NormalTok{,Y.with.X,Y.without.X))}
\NormalTok{df }\SpecialCharTok{\%\textgreater{}\%} \FunctionTok{group\_by}\NormalTok{(X) }\SpecialCharTok{\%\textgreater{}\%} \FunctionTok{summarize}\NormalTok{(}\AttributeTok{Y =} \FunctionTok{mean}\NormalTok{(Observed.Y))}
\end{Highlighting}
\end{Shaded}

\begin{verbatim}
## # A tibble: 2 x 2
##   X         Y
##   <lgl> <dbl>
## 1 FALSE 0.357
## 2 TRUE  1.83
\end{verbatim}

\begin{Shaded}
\begin{Highlighting}[]
\CommentTok{\#But if we properly model the process and compare apples to apples...}
\NormalTok{df }\SpecialCharTok{\%\textgreater{}\%} \FunctionTok{filter}\NormalTok{(}\FunctionTok{abs}\NormalTok{(Z}\FloatTok{{-}.7}\NormalTok{)}\SpecialCharTok{\textless{}}\NormalTok{.}\DecValTok{01}\NormalTok{) }\SpecialCharTok{\%\textgreater{}\%} \FunctionTok{group\_by}\NormalTok{(X) }\SpecialCharTok{\%\textgreater{}\%} \FunctionTok{summarize}\NormalTok{(}\AttributeTok{Y =} \FunctionTok{mean}\NormalTok{(Observed.Y))}
\end{Highlighting}
\end{Shaded}

\begin{verbatim}
## # A tibble: 2 x 2
##   X         Y
##   <lgl> <dbl>
## 1 FALSE 0.641
## 2 TRUE  1.70
\end{verbatim}
\end{frame}

\begin{frame}[fragile]{Identification}
\phantomsection\label{identification}
\begin{itemize}
\tightlist
\item
  We have ``identified'' a causal effect if \emph{the estimate that we
  generate gives us a causal effect}
\item
  In other words, \textbf{when we see the estimate, we can claim that
  it's isolating just the causal effect}
\item
  Simply looking at \texttt{lm(Y\textasciitilde{}X)} gives us the causal
  effect in the randomized-X case. \texttt{lm(Y\textasciitilde{}X)}
  \textbf{identifies} the effect of \(X\) on \(Y\)
\item
  But \texttt{lm(Y\textasciitilde{}X)} does \emph{not} give us the
  causal effect in the non-randomized case we did. In that case,
  \texttt{lm(Y\textasciitilde{}X)} does not \textbf{identify} the causal
  effect, but the apples-to-apples comparison we did \emph{does}
  identify the effect
\item
  Causal inference is all about figuring out \textbf{what calculation we
  need to do to identify that effect}
\end{itemize}
\end{frame}

\begin{frame}{Identification}
\phantomsection\label{identification-1}
\begin{itemize}
\tightlist
\item
  Identifying effects requires us to understand the \textbf{data
  generating process} (DGP)
\item
  And once we understand that DGP, knowing what calculations we need to
  do to isolate our effect
\item
  Often these will require taking some conditional values (controls)
\item
  Or \textbf{isolating the variation we want} in some othe rway
\end{itemize}
\end{frame}

\begin{frame}[fragile]{So!}
\phantomsection\label{so}
\begin{itemize}
\tightlist
\item
  So, as we move forward\ldots{}
\item
  We're going to be thinking about how to create models of the processes
  that generated the data
\item
  And, once we have those models, we'll figure out what methods we can
  use to generate plausible counterfactuals
\item
  Once we're really comparing apples to apples, we can figure out, using
  \emph{only data we can actually observe}, how things would be
  different if we reached in and changed \texttt{X}, and how \texttt{Y}
  would change as a result.
\end{itemize}
\end{frame}

\end{document}
